\documentclass[12pt]{report}			% Začátek dokumentu
\usepackage{MP}							% Import stylu

\author{Alexandr Bihun}
\title{Vizualizace významných algoritmů}
\date{14. února 2024}
\vedouci{Dr. rer. nat. Michal Kočer}
\place{V Českých Budějovicích}
\skolnirok{2023/2024}
\logo{\includegraphics[scale=1.25]{GJ8_logotyp}}

\begin{document}
	\mytitlepage						% Vygenerování titulní strany
	
	\prohlaseni{
		Prohlašuji, že jsem tuto práci vypracoval samostatně s vyznačením všech použitých pramenů.
	}	
	
	\abstrakt{
		Tato maturitní práce se zaměřuje na vystětlení chodu známých algoritmů v oblasti pathfindingu (vyhledávání cest), rovněž jako na jejich analýzu a příblížení jejich využití v opravdovém světě. Dále bude naznačeno, jak jsem implementoval za pomoci knihovny Pygame v jazyce Python uživatelsky přívětivou aplikaci pro vizualici těchto algoritmů, která umožňuje uživatelům hlubší porozumnění a poskytuje skutečný vhled na fuknci těchto algoritmů. % Abstrakt
	}{
		algoritmus, analýza algoritmů, hledání cest, grafy,  vizualizace, python, pygame					% Klíčová slova
	}
	
	\podekovani{
		Tady bude poděkování.						% Poděkování
	}
	
   {\tableofcontents\newpage}			% Obsah
	
%\addtocounter{page}{1}		% Posunutí countru stránek	
\pagenumbering{arabic}		% Číslování stránke arabskými číslicemi
	\chapter*{Úvod}
		Přesto, že si to většina lidí nejspíše neuvědumuje, využívají algoritmy na denním pořádku. blabla...	
		V této práci se zaměřím výhradně na algoritmy pro hledání cest	
	
	
	\part{Představení vybraných algoritmů}
	
		\chapter{O algoritmech obecně}
			
			\section{Trocha historie}
			
			\section{Reálné využítí}
				

		\chapter{Vsuvka z teorie grafů}
		
		\chapter{Analýza algoritmů}
			\section{Časová složitost}
			
			\section{Prostorová složitost}			
				
		\chapter{Algoritmy}
			\section{Prohledávání do hloubky}
			
			\section{Prohledávání do šířky}
			
			\section{Dijkstrův algoritmus}
			
			\section{Uspořádáné vyhledávání}
			
			\section{Algoritmus A*}
			
	
	\part{Implementace vizualizačního programu}

		\section{Výpisy použitých programů}




	\chapter*{Závěr}
	
		Tady bude závěr.
	
	\nocite{*}
    \printbibliography					% Vytvoří seznam literatury
	\addcontentsline{toc}{chapter}{Bibliografie}
    \printglossary[title={Zkratky}]		% Vytvoří seznam zkratek
    \listoffigures						% Vytvoří seznam obrázků
    \listoftables						% Vytvoří seznam tabulek

    \begin{appendices}
	\chapter{Příloha s kódem}	

	\end{appendices}
\end{document}
